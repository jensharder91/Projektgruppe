\documentclass{article}

\usepackage[ngerman]{babel}
\usepackage[utf8x]{inputenc}
\usepackage[T1]{fontenc}
\usepackage{lmodern}
\usepackage{graphicx} 
\usepackage{subfigure}
\usepackage{wrapfig}
\usepackage{adjustbox}
\usepackage{caption}
\usepackage{enumitem}
\usepackage {ulem}
%\usepackage{algorithmic}
\usepackage{algpseudocode}
\usepackage{geometry}
\geometry{a4paper, left=2.5cm, right=2.5cm, top=2.5cm, bottom=3cm}

\title{Bachelorarbeit\\
	Algorithmische Lösung des Potenzial-Problems\\
	auf Grundlage des Papers:\\
	"Capture of an Indruder by Mobile Agents"}
\author{Jens Harder}
\date{\today}


\begin{document}
	\maketitle
	\newpage
	
	\pagenumbering{arabic}.
	\tableofcontents
	
	\newpage
	
	\section{Einleitung}
Diese Bachelorarbeit ist aus einer Projektarbeit entstanden, welche im Semester 2014/15 stattgefunden hat.\\

In dieser Projektarbeit habe ich einen Algorithmus aus dem Paper "'Capture of an Intruder by Mobile Agents"' (BARRIÈRE L. et al. 2002) in einem Java-Applet visuell dargestellt und implementiert. Es handelt sich bei dem Algorithmus um eine Variante des sogenannten "'graph-searching problems"'.\\

Man kann sich das Problem anschaulich wie folgt vorstellen: Ein Einbrecher ist in das Netzwerk eingedrungen und soll nun von mobilen Agenten gesucht werden. Sowohl die Agenten, als auch der Einbrecher können sich nur auf den Kanten des Graphen bewegen. Der Algorithmus soll die minimale Anzahl der Agenten berechnen, sowie den Knoten angeben, wo die Agenten die Suche im Baum anfangen sollen (die sogenannte Homebase), ohne zu wissen, wo sich der Einbrecher befindet. Je nachdem welcher Knoten als Homebase ausgewählt wird, kann sich die benötigte Anzahl an Agenten evt. ändern.\\

Da dieses Problem NP-Vollständig auf auf allgemeinen Graphen ist, werden sowohl im Paper, als auch in dieser Bachelorarbeit, nur Bäume betrachtet. Dadurch ist es möglich, einen linearen Algorithmus anzugeben, in dem die minimale Anzahl an Agenten sowie die dazugehörige Homebase berechnet werden.\\

Eine wichtige Eigenschaft, die wir aufrecht erhalten wollen ist die Monotonie. Diese bedeutet in diesem Fall, dass ein Knoten, den wir bereits dekontaminiert haben (kontrolliert, dass der Einbrecher dort nicht ist), nicht mehr kontaminiert werden kann (der Einbrecher bekommt keine Möglichkeit mehr in diesen Knoten zu gelangen). Um diese Eigenschaft zu gewährleisten müssen wir sogenannte Wachen aufstellen, die dekontaminierte Teilbäume vor dem Einbrecher schützen. Die einzelnen Teilbäume werden von den Agenten sukzessive kontrolliert, bis wir den ganzen Baum dekontaminiert haben.\\
\\
\\

Das Ziel dieser Bachelorarbeit ist es, auf Grundlage des in "'Capture of an Intruder by Mobile Agents"' (BARRIÈRE L. et al. 2002) beschriebenen Algorithmus weitere Problemstellungen zu betrachten. Ich untersuche im folgenden, wie sich ein Potenzial auf den Algorithmus auswirkt. Das Potenzial gibt an, um wieviel eine oder mehrere Kanten(-gewichte) reduziert werden können. Der Algorithmus soll in der Lage sein, dieses Potenzial zu nutzen, um die minimale Anzahl an Agenten, die gebraucht werden, um den ganzen Baum zu dekontaminieren, zu denken. Je nachdem, welche Kanten durch das Potenzial reduziert werden, ändert sich die minimale Anzahl der Agenten unterschiedliche stark. Es sollen nun eine optimale Verteilung an Kanten angegeben werden, um die Agentenzahl möglichst stark zu reduzieren.\\



	
	\section{allg Problemstellung}
Im folgen werde ich zunächst den Algorithmus aus dem Paper beschreiben und erklären, sowie im Anschluss ein paar Modifikationen, mit denen ich in dieser Bachelorarbeit weiterarbeiten werde.



\subsection{orginaler Algo}
Um die Anzahl der Agenten zu ermitteln, schicken sich die Baumknoten gegenseitig Nachrichten, mit der Information, wie viele Agenten mindestens benötigt werden, um einen bestimmten Teilbaum zu dekontaminieren.
\\
Als erstes berechnet der Algorithmus alle von allen Knoten ihre Knotengewichte. Dieses ergibt sich jeweils aus dem maximalen Kantengewicht aller zu diesem Knoten führenden Kanten. Also für das Knotengewicht vom Knoten x gilt: $\omega(x) = max_{e} \omega(e)$, für jede Kante e inzident zu x.
\\
\\
Nun werden die Nachrichten versendet, für die es zwei Fälle gibt:

\begin{enumerate}
		
	
	\item Fall:
	
		\begin{minipage}{0.55\textwidth} 
			Der aktuelle Knoten x hat bereits schon mindestens n-1 Nachrichten erhalten, wobei n die Anzahl der inzidenten Knoten ist.\\
			
			Um eine Nachricht an den Nachbarknoten y zu senden, nimmt man die 2 größten Nachrichten die bei x angekommen sind (die Nachricht von y wird ignoriert!). Mit diesen beiden angekommenen Nachrichten $l_{1} \ge l_{2}$ sowie dem Knotengewicht $\omega(x)$ wird die Nachricht $\lambda_{y}$ an y wie folgt berechnet:\\
			$\lambda_{y} = max\{l_{1},  l_{2} + \omega(x)\}$
		\end{minipage}
		\hfill
		\begin{minipage}{0.35\textwidth}
			
			\includegraphics[width=\textwidth]{bilder/abb_paper_n-1knoten.png}
			\captionof{figure}{Die neue Nachricht (blau) von Knoten x zu Knoten y ist 6, da $l_{2} + \omega(x)$ (grün) größer ist als $l_{1}$ (gelb).}
		\end{minipage}

		
		
	\item Fall:
		
		\begin{minipage}{0.55\textwidth} 
			Zu beginn des Algorithmus, oder wenn der andere Fall nicht mehr auftritt, sendet ein beliebiger Blattknoten seine Nachricht an den eigenen Nachbarn.\\
			
			Die zu sendende Nachricht $\lambda_{y}$ vom Blatt x an seinen Nachbarknoten y ist dabei nur das eigene Gewicht:\\
			$\lambda_{y} = \omega(x)$
		\end{minipage}
		\hfill
		\begin{minipage}{0.35\textwidth}
						
			\includegraphics[width=\textwidth]{bilder/abb_blattknoten.png}
			\captionof{figure}{Das Knotengewicht des Blattknotens x (grün) bestimmt die Nachricht (blau) zum Nachbarknoten y.}
		\end{minipage}
		

\end{enumerate}


\subsection{neue interpreatation}
//wieso?

		fzw aegf wfgeewg fzw aegf wfgeewg fzw aegf wfgeewg fzw aegf wfgeewg fzw aegf wfgeewg fzw aegf wfgeewg fzw aegf wfgeewg fzw aegf wfgeewg fzw aegf wfgeewg fzw aegf wfgeewg fzw aegf wfgeewg fzw aegf wfgeewg fzw aegf wfgeewg fzw aegf wfgeewg fzw aegf wfgeewg fzw aegf wfgeewg fzw aegf wfgeewg fzw aegf wfgeewg fzw aegf wfgeewg fzw aegf wfgeewg fzw aegf wfgeewg fzw aegf wfgeewg fzw aegf wfgeewg fzw aegf wfgeewg fzw aegf wfgeewg fzw aegf wfgeewg fzw aegf wfgeewg fzw aegf wfgeewg fzw aegf wfgeewg fzw aegf wfgeewg fzw aegf wfgeewg fzw aegf wfgeewg fzw aegf wfgeewg fzw aegf wfgeewg fzw aegf wfgeewg fzw aegf wfgeewg fzw aegf wfgeewg fzw aegf wfgeewg fzw aegf wfgeewg fzw aegf wfgeewg fzw aegf wfgeewg fzw aegf wfgeewg fzw aegf wfgeewg fzw aegf wfgeewg 


		\begin{wrapfigure}{r}{0.5\textwidth}
			\begin{center}
				\includegraphics[width=0.48\textwidth]{bilder/abb_blattknoten.png}
			\end{center}
			\caption{A gull}
		\end{wrapfigure}
		
		
		
		
		
		fzw aegf wfgeewg fzw aegf wfgeewg fzw aegf wfgeewg fzw aegf wfgeewg fzw aegf wfgeewg fzw aegf wfgeewg fzw aegf wfgeewg fzw aegf wfgeewg fzw aegf wfgeewg fzw aegf wfgeewg fzw aegf wfgeewg fzw aegf wfgeewg fzw aegf wfgeewg fzw aegf wfgeewg fzw aegf wfgeewg fzw aegf wfgeewg fzw aegf wfgeewg fzw aegf wfgeewg fzw aegf wfgeewg fzw aegf wfgeewg fzw aegf wfgeewg fzw aegf wfgeewg fzw aegf wfgeewg fzw aegf wfgeewg fzw aegf wfgeewg fzw aegf wfgeewg fzw aegf wfgeewg fzw aegf wfgeewg fzw aegf wfgeewg fzw aegf wfgeewg fzw aegf wfgeewg fzw aegf wfgeewg wfgeewg fzw aegf wfgeewg fzw aegf wfgeewg fzw aegf wfgeewg fzw aegf wfgeewg fzw aegf wfgeewg fzw aegf wfgeewg fzw aegf wfgeewg fzw aegf wfgeewg fzw aegf wfgeewg fzw aegf wfgeewg fzw aegf wfgeewg fzw aegf wfgeewg fzwfgeewg fzw aegf wfgeewg fzw aegf wfgeewg fzw aegf wfgeewg fzw aegf wfgeewg fzw aegf wfgeewg fzw aegf wfgeewg fzw aegf wfgeewg fzw aegf wfgeewg fzw aegf wfgeewg fzw aegf wfgeewg fzw aegf wfgeewg fzw aegf wfgeewg fzwfgeewg fzw aegf wfgeewg fzw aegf wfgeewg fzw aegf wfgeewg fzw aegf wfgeewg fzw aegf wfgeewg fzw aegf wfgeewg fzw aegf wfgeewg fzw aegf wfgeewg fzw aegf wfgeewg fzw aegf wfgeewg fzw aegf wfgeewg fzw aegf wfgeewg fzwfgeewg fzw aegf wfgeewg fzw aegf wfgeewg fzw aegf wfgeewg fzw aegf wfgeewg fzw aegf wfgeewg fzw aegf wfgeewg fzw aegf wfgeewg fzw aegf wfgeewg fzw aegf wfgeewg fzw aegf wfgeewg fzw aegf wfgeewg fzw aegf wfgeewg fz

		
	

	\section{Das Potenzial-Problem}

\subsection{Spezialfall k = 1}
//algo angeben, wieso ist er liniear? wieso korrekt?
\\
\\
Jeder Knoten berechnet im Algorithmus von BARRIÈRE L. et al. die minimale Anzahl von Agenten (siehe oben) und es lässt sich schnell überlegen, dass die Höhe der Agentenanzahl an jedem Knoten von bestimmten Kanten(-gewichten) abhängt.
\\
\\
Die Idee, um das Potenzial-Problem für den Spezialfall k = 1 zu lösen, ist nun, die Abhängigkeiten zu speichern, welche Kante(n) die Agentenzahl in jedem Knoten beeinflusst. Dazu protokollieren wir im Algorithmus, wie jede berechnete Nachricht zustande kommt, also von welchen Kanten sie abhängt.
\\
\\ 
Wie man in der modifizierten Variante (siehe oben) sehen kann, gibt es drei verschiedene Möglichkeiten, wie eine Nachricht 	$\lambda_{y}$ entstehen kann:

\begin{enumerate}[label=\alph*)]
	
	\item aus den beiden größten Kanten $\lambda_{y} \gets edge_{1} + edge_{2}$
	
	\item aus dem Knotengewicht $\lambda_{y} \gets \omega(x)$
	
	\item aus der größten angekommenen Nachricht $\lambda_{y} \gets l_{1}$

\end{enumerate}

Je nachdem, durch welchen Fall eine Nachricht $\lambda_{y}$ berechnet wird, kommen andere Kanten in Frage, die wir protokollieren müssen.
Allerdings gilt für alle Fälle, dass wir uns maximal zwei Kanten merken müssen, da wir ansonsten die minimale Agentenzahl nicht verringern können. Gibt es mehr als zwei Kanten, so merken wir uns nur die Information, dass es keinen (zwei) eindeutigen Kanten gab bis hierhin (wir setzen einen "flag").
\\
\\
//TODO ?! beweis, dass nur 2 Kanten in Frage kommen
\\
\\
Im folgenden werde ich bei allen drei Fällen beschreiben, welche Kanten unter welcher Bedingung protokolliert werden. Protokollieren bedeutet einfach, dass wir bei der Berechnung der Nachricht $\lambda_{y}$ von x nach y  in die Nachricht mit reinschreiben, welche Kanten diese Nachricht bestimmt haben, bzw. wir protokollieren einen "flag", falls dies mehr als zwei Kanten sind.

\begin{enumerate}[label=\alph*)]
	
	\item in diesem Standardfall wird die Nachricht aus den beiden größten Kanten berechnet (wie oben beschrieben). Allerdings müssen wir kontrollieren, ob diese zwei Kanten eindeutig sind, oder ob es evtl. mehrere gleich große Kanten gibt. Ist die Wahl der Kanten nicht eindeutig, müssen wir dies bei der Protokollierung mit berücksichtigen:\\
	
		\begin{algorithmic}
			\If {$edge_{1} == edge_{3}$}
			\State \uline{protokolliere} "flag"
			\State//es gibt drei gleichgroße Kanten
			\ElsIf {$edge_{2} == edge_{3}$}
			\State \uline{protokolliere} $edge_{1}$
			\State//die maximale Kante ist eindeutig, die zweit größte nicht
			\Else
			\State \uline{protokolliere} $edge_{1}$ und $edge_{2}$
			\State//sowohl die größte, als auch die zweit größte Kante ist eindeutig
			\EndIf
		\end{algorithmic}
	
	\item Dieser Fall tritt nur auf, wenn $\omega(x) \geq edge_{1}+edge_{2}$. Da aber $\omega(x)$ so definiert ist, dass es den Wert des größten Kantengewicht inzident zu x hat, muss die Kante ausschlaggebend sein, über die die Nachricht verschickt wird. 
	\\
	Diese Kante zwischen x und y ist für den Nachrichtenwert also entscheidend und wird somit in der Nachricht: \uline{Protokolliere} Kante zwischen x und y.
	
	\item In diesem Fall bestimmt die größte ankommende Nachricht $l_{1}$ die neu berechnete Nachricht. Da keine weitere Kante mehr Einfluss genommen hat, übernehmen wir die protokollierten Kanten (oder die "flag") aus $l_{1}$ für $\lambda_{y}$: \uline{Protokolliere} das gleiche wie $l_{1}$
	
\end{enumerate}

//TODO alle einzelnen Fälle noch genauer erklären
\\
\\

Man muss noch beachten, dass mehrere Fälle gleichzeitig auftreten können. Passiert dies, muss man einen weiteren Test durchführen, ob die verschiedenen Fälle unterschiedlich protokollieren würden.
\\
Würden die Fälle verschieden protokollieren, so \uline{protokolliere} "flag", da das Potenzial an dieser Stelle nicht genutzt werden kann, ansonsten behalte die \uline{Protokollierung} bei.

\subsection{Potenzial auf einer Kante mit k > 1}

Um das Problem des Potenzials etwas zu erweitern, betrachte ich im folgenden Abschnitt Potenziale größer 1, allerdings mit der Einschränkung, dass man das gesamte Potenzial k nur auf einer Kante einsetzen darf.

Es fällt zunächst auf, dass man nicht garantieren kann, dass sich die Anzahl der benötigten Agenten auf allen Bäumen reduzieren lässt. Hierbei spielt es auch keine Rolle, wie groß das Potenzial k ist, da allein die Eigenschaft, dass man das Potenzial nur auf einer Kante einsetzen kann, genügt, um ein Gegenbeispiel zu finden:

Da das Potenzial k beliebig groß ist, kann man eine Kante auf Kantengewicht 1 setzen, man dadurch das größtmögliche Potenzial ausnutzt. Trotzdem ist es nicht möglich bei folgendem Baum die Anzahl der Agenten zu reduzieren:

\begin{figure}[h]
	\subfigure[alle Kanten haben Gewicht 4. Alle Knoten benötigen mindestens 8 Agenten.]{\includegraphics[width=0.49\textwidth]{bilder/abb1.png}} 
	\hfill
	\subfigure[eine Kante wurde auf Gewicht 1 geduziert, alle anderen haben weiterhin Gewicht 4. Alle Knoten benötigen trotzdem mindestens 8 Agenten.]{\includegraphics[width=0.49\textwidth]{bilder/abb2.png}} 
	\caption{Beispiel, dass Verringerung auf einer Kante nicht zu einer Verringerung der notwendigen Agenten führen muss} 
\end{figure} 

TODO:\\
//überlegung: woran sieht man, ob man die Agentenanzahl verringern kann?\\
//algorithmus angeben, welche kante ausgesucht wird?\\

\subsection{Potenzial verteilen mit k > 1}
	
	\section{Kooperative Gruppe}
	// wird das behandelt???
	
	\section{Fazit}
	
	
\end{document}
