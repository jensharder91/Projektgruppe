\section{Einleitung}
Diese Bachelorarbeit ist aus einer Projektarbeit entstanden, welche im Semester 2014/15 stattgefunden hat.
\\
\\
In dieser Projektarbeit habe ich einen Algorithmus aus dem Paper "'Capture of an Intruder by Mobile Agents"' (BARRIÈRE L. et al. 2002) in einem Java-Applet visuell dargestellt und implementiert. Es handelt sich bei dem Algorithmus um eine Variante des sogenannten "'graph-searching problems"'.
\\
\\
Man kann sich das Problem anschaulich wie folgt vorstellen: Ein Einbrecher ist in das Netzwerk eingedrungen und soll nun von mobilen Agenten gesucht werden. Sowohl die Agenten, als auch der Einbrecher können sich nur auf den Kanten des Graphen bewegen. Der Algorithmus soll die minimale Anzahl der Agenten berechnen, sowie den Knoten angeben, wo die Agenten die Suche im Baum anfangen sollen (die sogenannte Homebase), ohne zu wissen, wo sich der Einbrecher befindet. Je nachdem welcher Knoten als Homebase ausgewählt wird, kann sich die benötigte Anzahl an Agenten evtl. ändern.
\\
In der im Paper beschriebenen Variante des "'graph-searching problems"' hat jede Kante ein Gewicht $\geq 1$. Dieses Gewicht sagt aus, wieviele Agenten mindestens über eine Kante laufen müssen, um diese zu dekontaminieren. Anschaulich kann man sich die Kante als Gang in einem Haus vorstellen. Je nachdem, wie groß und verwinkelt dieser Gang ist, braucht man verschieden viele Agenten, um diesen Gang zu dekontaminieren, bzw. um alle Ecken und Winkel in diesem Gang zu kontrollieren, ob sich der Einbrecher dort irgendwo versteckt hält.
\\
\\
Da dieses Problem NP-vollständig auf allgemeinen Graphen ist, werden sowohl im Paper, als auch in dieser Bachelorarbeit, nur Bäume betrachtet. Dadurch ist es möglich, einen linearen Algorithmus anzugeben, in dem die minimale Anzahl an Agenten sowie die dazugehörige Homebase berechnet werden.
\\
\\
Eine wichtige Eigenschaft, die wir aufrecht erhalten wollen, ist die Monotonie. Diese bedeutet in diesem Fall, dass ein bereits dekontaminierter Knoten nicht mehr kontaminiert werden kann. Um diese Eigenschaft zu gewährleisten müssen wir sogenannte Wachen aufstellen, die dekontaminierte Teilbäume vor dem Einbrecher schützen. Die Anzahl der benötigten Wachen hängt von der größten inzidenten Kante ab, die zu einem kontaminierten Teilbaum führt. Gibt es keinen solchen Teilbaum, wird auch keine Wache benötigt. Die Agenten dekontaminieren einzelne Teilbäume sukzessive, bis der gesamte Baum bearbeitet wurde.
\\
\\
Das Ziel dieser Bachelorarbeit ist es, auf Grundlage des in "'Capture of an Intruder by Mobile Agents"' (BARRIÈRE L. et al. 2002) beschriebenen Algorithmus, weitere Problemstellungen zu betrachten. Ich untersuche im folgenden, wie sich ein sogenanntes Potenzial auf den Algorithmus auswirkt. Das Potenzial gibt an, um wie viel eine oder mehrere Kanten(-gewichte) reduziert werden können. Ich entwickle nun einen Algorithmus für verschiedene Varianten des Potenzial-Problems. Der Algorithmus kann dabei ein Potenzial nutzt um einzelne Kantengewichte zu verringern. Dadurch sollen die für die Dekontaminierung benötigte Anzahl an Agenten verringert werden. Allerdings ändert sich die minimale Anzahl der Agenten unterschiedliche stark, je nachdem, welche Kanten durch das Potenzial reduziert werden, weshalb der Algorithmus eine optimale Aufteilung des Potenzials bestimmen soll, um die Agentenzahl möglichst stark zu senken.
\\

